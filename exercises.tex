% Notes and exercises on An Introduction to Mathematical Cryptography
\documentclass[letterpaper,12pt]{article}
\usepackage{amsmath,amssymb,amsthm,enumitem,fourier}

% Definitions
\newcommand{\Z}{\mathbb{Z}}
\newcommand{\F}{\mathbb{F}}
\newcommand{\Fp}{\F_{p}}
\newcommand{\Fps}{\Fp^{*}}

\newcommand{\divides}{\mathrel{|}}
\newcommand{\xor}{\oplus}

\renewcommand{\times}{\mathop{\cdot}}

\newcommand{\floor}[1]{\left\lfloor{#1}\right\rfloor}

\theoremstyle{definition}
\newtheorem*{exer}{Exercise}

% Meta
\title{\textit{An Introduction to Mathematical Cryptography}\\Notes and Exercises}
\author{John Peloquin}
\date{}

\begin{document}
\maketitle

\section*{Chapter~1}
\begin{exer}[1.5]
Suppose \(A\)~is an alphabet of \(n\)~letters.
\begin{enumerate}[itemsep=0pt]
\item[(a)] There are \(n!\)~simple substitution ciphers on~\(A\).
\item[(b)]
Define the function~\(!n\) recursively by\footnote{A more efficient recursive definition for \(n>1\) is given by \(!n=(n-1)[!(n-1)+!(n-2)]\).}
\[!n=\begin{cases}
1&\text{if }n=0\\
0&\text{if }n=1\\
n!-\sum_{k=1}^n\binom{n}{k}\times{!(n-k)}&\text{if }n>1
\end{cases}\]
\begin{enumerate}[itemsep=0pt]
\item[(i)] There are \(!n\)~simple substitution ciphers on~\(A\) leaving no letters fixed.
\item[(ii)] There are \(n!-{!n}\) simple substitution ciphers on~\(A\) leaving at least one letter fixed.
\item[(iii)] There are \(n\times{!(n-1)}\) simple substitution ciphers on~\(A\) leaving exactly one letter fixed.
\item[(iv)] There are \(n!-{!n}-n\times{!(n-1)}\) simple substitution chipers on~\(A\) leaving at least two letters fixed.
\end{enumerate}
\end{enumerate}
\end{exer}
\begin{proof}
For~(a), note that a simple substitution cipher on~\(A\) is just a permutation of~\(A\), and there are \(n!\)~permutations of~\(A\).

For~(b), we prove~(i) by induction on~\(n\). Cases \(n=0,1\) are trivial. Suppose \(n>1\) and the result is true for all \(m<n\). Let \(\varphi_k\)~be the number of permutations of~\(A\) fixing exactly \(k\)~elements, for \(1\le k\le n\). Clearly the desired number is
\[n!-\sum_{k=1}^n\varphi_k\]
To compute~\(\varphi_k\), note that in order to permute the elements of~\(A\) while fixing \(k\)~elements, we must choose \(k\)~elements to fix and then permute the remaining \(n-k\) elements without fixing any of them. The number of possible ways to choose \(k\)~elements from~\(A\) is just \(\binom{n}{k}\). By induction, the number of ways to permute the remaining \(n-k\) elements without fixing any of them is \(!(n-k)\). So \(\varphi_k=\binom{n}{k}\times{!(n-k)}\).

Now (ii)~follows trivially from (a)~and~(i); (iii)~follows from the proof of~(ii) by taking~\(\varphi_1\); and (iv)~follows trivially from (a), (i), and~(iii).
\end{proof}
\noindent For example, when \(n=4\), there are \(24\)~simple substitution ciphers, \(9\)~of which leave no letters fixed, \(15\)~of which leave at least one letter fixed, \(8\)~of which leave exactly one letter fixed, and \(7\)~of which leave at least two letters fixed.

\begin{exer}[1.11]
Let \(a\)~and~\(b\) be positive integers.
\begin{enumerate}[itemsep=0pt]
\item[(a)] If \(u\)~and~\(v\) are integers with \(au+bv=1\), then \(\gcd(a,b)=1\).
\item[(b)] If \(u\)~and~\(v\) are integers with \(au+bv=6\), it is not necessarily true that \(\gcd(a,b)=6\), but \(\gcd(a,b)\) will be a divisor of~\(6\).
\item[(c)] If \((u_1,v_1)\) and \((u_2,v_2)\) are solutions of \(au+bv=1\), then \(a\)~divides \(v_2-v_1\) and \(b\)~divides \(u_2-u_1\).
\item[(d)] If \(g=\gcd(a,b)\) and \((u_0,v_0)\) is a solution to \(au+bv=g\), then each other solution has the form
\[u=u_0+\frac{kb}{g}\qquad v=v_0-\frac{ka}{g}\]
for some integer~\(k\).
\end{enumerate}
\end{exer}
\begin{proof}
\begin{enumerate}[itemsep=0pt]
\item[(a)] Trivially \(1\divides a\) and \(1\divides b\). If \(d\divides a\) and \(d\divides b\), then \(d\divides au\) and \(d\divides bv\), so \(d\divides au+bv=1\).
\item[(b)] Let \(a=b=3\) and \(u=v=1\). Then \(au+bv=6\ne 3=\gcd(a,b)\). The second claim follows from the proof of~(a).
\item[(c)] We have
\begin{align*}
au_1+bv_1&=au_2+bv_2\\
au_1-au_2&=bv_2-bv_1\\
a(u_1-u_2)&=b(v_2-v_1)
\end{align*}
Now \(a\divides b(v_2-v_1)\), but \(\gcd(a,b)=1\) by~(a), so \(a\divides v_2-v_1\). Similarly \(b\divides u_2-u_1\).
\item[(d)] If \((u,v)\) is another solution, then
\[au_0+bv_0=au+bv=g\]
Dividing by~\(g\), this becomes
\[\frac{a}{g}u_0+\frac{b}{g}v_0=\frac{a}{g}u+\frac{b}{g}v=1\]
Let \(a'=a/g\) and \(b'=b/g\). By the proof of~(c), there exists~\(k\) such that \(ka'=v_0-v\) and \(kb'=u-u_0\), so
\[u=u_0+kb'=u_0+\frac{kb}{g}\qquad v=v_0-ka'=v_0-\frac{ka}{g}\qedhere\]
\end{enumerate}
\end{proof}

\begin{exer}[1.21]
Let \(m>1\).
\begin{enumerate}[itemsep=0pt]
\item[(a)] If \(m\)~is odd, then \((m+1)/2=2^{-1}\bmod m\).
\item[(b)] If \(b>0\) and \(m\equiv 1\bmod b\), then
\[\frac{(b-1)(m-1)}{b}+1=b^{-1}\bmod m\]
\end{enumerate}
\end{exer}
\begin{proof}
Note (a)~follows from~(b) by taking \(b=2\), so it is sufficient to prove~(b).

Since \(b\divides m-1\), the quantity on the left of the equation is an integer, and since \(m>1\),
\begin{align*}
0&\le (b-1)(m-1)<b(m-1)\\
0&\le\frac{(b-1)(m-1)}{b}<m-1\\
1&\le\frac{(b-1)(m-1)}{b}+1\le m-1
\end{align*}
Finally,
\[b\Bigl[\frac{(b-1)(m-1)}{b}+1\Bigr]=(b-1)(m-1)+b=(b-1)m+1\equiv 1\bmod m\qedhere\]
\end{proof}

\begin{exer}[1.26]
There are infinitely many primes.
\end{exer}
\begin{proof}
Suppose there are only finitely many primes, say \(p_1,\ldots,p_k\). Let
\[N=p_1\cdots p_k+1\]
By the fundamental theorem of arithmetic, there exists a prime~\(p\) with \(p\divides N\), so \(N\equiv 0\bmod p\). But by assumption, \(p=p_i\) for some \(1\le i\le k\), so \(p\divides p_1\cdots p_k\) and hence \(N\equiv 1\bmod p\), a contradiction.
\end{proof}

\begin{exer}[1.33]
If \(p\)~is prime, \(q=(p-1)/2\) is also prime, \(0<g<p\), \(g\not\equiv\pm 1\bmod p\), and \(g^q\not\equiv 1\bmod p\), then \(g\)~is a primitive root mod~\(p\).
\end{exer}
\begin{proof}
Let \(n\)~be the order of~\(g\) mod~\(p\). We claim \(n=p-1\).

By assumption, \(g^q\not\equiv 1\bmod p\), but by Fermat's little theorem,
\[(g^q)^2=g^{2q}=g^{p-1}\equiv 1\bmod p\]
Therefore \(g^q\)~has order~\(2\) mod~\(p\), and hence \(2\divides n\) by Lagrange's theorem. If \(n=2\), then \(g^2\equiv 1\bmod p\), hence \(g\equiv\pm 1\bmod p\), contrary to our assumption. So we must have \(n>2\).

Now \(n\divides p-1\), so \(p-1=nk=2jk\) for some integers \(j\)~and~\(k\) with \(j>1\). But since \(jk=(p-1)/2=q\) is prime, this means \(k=1\) and hence \(n=p-1\) as desired.
\end{proof}

\begin{exer}[1.46]
The XOR cipher defined on bit strings by
\[e_k(m)=k\xor m\qquad\text{and}\qquad d_k(c)=k\xor c\]
is not secure against a chosen plaintext attack.
\end{exer}
\begin{proof}
If the pair \(m\)~and \(c=k\xor m\) are known, then \(k\)~is easily recovered as
\[c\xor m=(k\xor m)\xor m=k\xor(m\xor m)=k\xor 0=k\qedhere\]
\end{proof}
\noindent For example, working with 16-bit strings, if \(m=0010010000101100\) and \(c=1001010001010111\), then \(k=1011000001111011\).

\section*{Chapter 2}
\begin{exer}[2.3]
Let \(g\)~be a primitive root modulo~\(p\). Define
\[\log_g:\Fps\to\Z/(p-1)\Z\]
as follows: for \(h\in\Fps\), \(\log_g(h)=x\) if and only if \(0\le x<p-1\) and \(g^x=h\). Then \(\log_g\)~witnesses an isomorphism from~\(\Fps\) to~\(\Z/(p-1)\Z\).
\end{exer}
\begin{proof}
First observe that \(\log_g\)~is well defined. Since \(g\)~is a primitive root mod~\(p\), for each \(h\in\Fps\) there exists at least one \(0\le x<p-1\) such that \(g^x=h\). If \(0\le x,y<p-1\) and \(g^x=h=g^y\), then \(g^{x-y}=1\), so \(p-1\divides x-y\) and hence \(x=y\).

Observe also that \(\log_g\)~is injective. If \(h_1,h_2\in\Fps\) and \(\log_g h_1=x=\log_g h_2\), then \(h_1=g^x=h_2\). Since \(\Fps\)~and~\(\Z/(p-1)\Z\) both have order~\(p-1\), this this also shows that \(\log_g\)~is surjective, and hence bijective.

If \(h_1,h_2\in\Fps\), we claim that
\[\log_g(h_1h_2)\equiv\log_g h_1+\log_g h_2\pmod{p-1}\]
Indeed, let \(x_1=\log_g h_1\) and \(x_2=\log_g h_2\), so \(h_1=g^{x_1}\) and \(h_2=g^{x_2}\). Then \(h_1h_2=g^{x_1}g^{x_2}=g^{x_1+x_2}\), so \(\log_g(h_1h_2)\equiv x_1+x_2\bmod{p-1}\) as claimed. This shows that \(\log_g\)~is a homomorphism, and since it is bijective, an isomorphism.
\end{proof}
\noindent Note by induction and the fact that \(\log_g(h^{-1})\equiv-\log_g(h)\bmod{p-1}\) for all \(h\in\Fps\), it also follows that
\[\log_g(h^n)\equiv n\log_g h\pmod{p-1}\]
for all \(h\in\Fps\) and \(n\in\Z\).

\begin{exer}[2.4]
Using the brute force algorithm of computing powers~\(g^x\bmod p\) manually for \(x=0,1,2,\ldots\) we obtain the following discrete logarithms:
\begin{enumerate}[itemsep=0pt]
\item[(a)] \(\log_{2}13=7\bmod{23}\)
\item[(b)] \(\log_{10}22=11\bmod{47}\)
\item[(c)] \(\log_{627}608=18\bmod{941}\)
\end{enumerate}
\end{exer}

\begin{exer}[2.5]
Let \(p\)~be an odd prime and \(g\)~be a primitive root modulo~\(p\). Then \(a\)~has a square root modulo~\(p\) if and only if \(\log_g a\)~is even.
\end{exer}
\begin{proof}
Since \(\log_g\)~is a homomorphism (Exercise~2.3), we know that
\[\log_g(x^2)\equiv 2\log_g(x)\pmod{p-1}\]
for all \(x\in\Fps\), from which the desired result follows immediately.
\end{proof}

\begin{exer}[2.9]
A Diffie-Hellman oracle can be used to decrypt arbitrary ElGamal ciphertexts.
\end{exer}
\begin{proof}
Let \(p\)~and~\(g\) be known parameters for ElGamal encryption. Given an ElGamal ciphertext \((c_1,c_2)\), we know that \(c_1=g^k\bmod p\) for some ephemeral key~\(k\) and \(c_2=mA^k=m(g^a)^k=mg^{ak}\bmod p\) for some plaintext message~\(m\) and public key \(A=g^a\bmod p\) with corresponding private key~\(a\). Now given \(A=g^a\) and \(c_2=g^k\), a Diffie-Hellman oracle provides us with \(g^{ak}\), with which we can easily compute \(m=c_2(g^{ak})^{-1}\).
\end{proof}
\noindent This result shows that decrypting arbitrary ElGamal ciphertexts is no harder than the Diffie-Hellan problem. Together with the converse result (Proposition~2.10), this result shows that the difficulty of decrypting arbitrary ElGamal ciphertexts is equal to that of the Diffie-Hellman problem.

\begin{exer}[2.17]
Using the Shanks algorithm, we obtain the following discrete logarithms:
\begin{enumerate}[itemsep=0pt]
\item[(a)] \(\log_{11}21=37\bmod{71}\)

First we compute \(n=\floor{\sqrt{71}}+1=8+1=9\). For \(i=0,\ldots,9\) we compute values~\(11^i\bmod{71}\) to obtain the following list:
\begin{center}
\begin{tabular}{|c|c|c|c|c|c|c|c|c|c|c|}
\hline
\(i\)&0&1&2&3&4&5&6&7&8&9\\
\hline
\(11^i\bmod{71}\)&1&11&50&53&15&23&40&14&12&61\\
\hline
\end{tabular}
\end{center}
Now we compute \(u=11^{-n}=11^{-9}=(11^9)^{-1}=61^{-1}=7\bmod{71}\). For \(j=0,\ldots,9\) we compute values~\(21\cdot 7^j\bmod{71}\) until we find a match:
\begin{center}
\begin{tabular}{|c|c|c|c|c|c|}
\hline
\(j\)&0&1&2&3&4\\
\hline
\(21\cdot 7^j\bmod{71}\)&21&5&35&32&11\\
\hline
\end{tabular}
\end{center}
We find a match~\(11\) at \(i=1\) and \(j=4\), so \(x=i+jn=1+4\cdot 9=37\) is the logarithm. Indeed, \(11^{37}\equiv 21\bmod{71}\).
\end{enumerate}
\end{exer}

\begin{exer}[2.18]
Using the Chinese remainder algorithm, we obtain solutions to the following systems of congruences:
\begin{enumerate}[itemsep=0pt]
\item[(a)] \(x\equiv 3\bmod 7\quad\text{and}\quad x\equiv 4\bmod 9\).

We start with the solution \(x=3\) to the first congruence. To find a solution to both congruences, we must find a solution to the second congruence which lies in the solution space of the first, which is \(x+7y\) for \(y\in\Z\). Such a solution must satisfy
\begin{align*}
3+7y&\equiv 4\pmod 9&&\\
7y&\equiv 1\pmod 9&&\\
y&\equiv 4\pmod 9&&\text{since }7^{-1}\equiv 4\bmod 9
\end{align*}
We take \(y=4\) to yield the solution \(3+7\cdot 4=31\). Indeed, \(31\equiv 3\bmod 7\) and \(31\equiv 4\bmod 9\) as desired.

\item[(d)] \(x\equiv 5\bmod 9\quad\text{and}\quad x\equiv 6\bmod{10}\quad\text{and}\quad x\equiv 7\bmod{11}\).

We start with the solution \(x=5\) to the first congruence. Proceeding as in~(a), we obtain a solution to the first and second congruences by solving
\begin{align*}
5+9y&\equiv 6\pmod{10}&&\\
9y&\equiv 1\pmod{10}&&\\
y&\equiv 9\pmod{10}&&\text{since }9^{-1}\equiv 9\bmod{10}
\end{align*}
We take \(y=9\) to yield the solution \(5+9\cdot 9=86\). Indeed, \(86\equiv 5\bmod 9\) and \(86\equiv 6\bmod{10}\) as desired.

To find a solution to all three congruences, we must find a solution to the third congruence which lies in the \emph{intersection} of the solution spaces of the first two, which is \(86+9\cdot 10z\) for \(y\in\Z\). Such a solution must satisfy
\begin{align*}
86+90z&\equiv 7\pmod{11}&&\\
9+2z&\equiv 7\pmod{11}&&\\
2z&\equiv 9\pmod{11}&&\\
z&\equiv 10\pmod{11}&&\text{since }2^{-1}\equiv 6\bmod{11}
\end{align*}
We take \(z=10\) to yield the solution \(86+90\cdot 10=986\). Indeed, \(986\equiv 5\bmod 9\), \(986\equiv 6\bmod 10\), and \(986\equiv 7\bmod 11\).
\end{enumerate}
\end{exer}

\begin{exer}[2.25]
Let \(p\)~and~\(q\) be distinct odd primes and let \(n=pq\).
\begin{enumerate}[itemsep=0pt]
\item[(a)] If \(\gcd(a,n)=1\) and \(a\)~has a square root modulo~\(n\), then \(a\)~has four square roots modulo~\(n\).
\end{enumerate}
\end{exer}
\begin{proof}
\begin{enumerate}[itemsep=0pt]
\item[(a)] Let \(r\)~be a square root of~\(a\) modulo~\(n\). Then \(x\)~is a square root of~\(a\) modulo~\(n\) if and only if \(x^2\equiv a\equiv r^2\bmod n\), which is true if and only if \(x^2\equiv r^2\bmod p\) and \(x^2\equiv r^2\bmod q\), which is true if and only if \(x\equiv\pm r\bmod p\) and \(x\equiv\pm r\bmod q\), since \(p\)~and~\(q\) are distinct primes.

This implies that the square roots of~\(a\) modulo~\(n\) are just the solutions to the following systems of congruences:
\begin{align}
x&\equiv r\pmod p&&\text{and}&x&\equiv r\pmod q\\
x&\equiv r\pmod p&&\text{and}&x&\equiv -r\pmod q\\
x&\equiv -r\pmod p&&\text{and}&x&\equiv r\pmod q\\
x&\equiv -r\pmod p&&\text{and}&x&\equiv -r\pmod q
\end{align}
Since \(p\ne q\), the Chinese remainder theorem guarantees that for each one of these systems, there is a unique solution modulo \(n=pq\). This implies that there are \emph{at most} four square roots of~\(a\) modulo~\(n\).

Now \(r\)~is one of the roots, and clearly so is~\(-r\). Note \(r\)~is also a unit since \(a\)~is a unit (because \(\gcd(a,n)=1\)). If \(r\equiv -r\bmod n\), then \(1\equiv -1\bmod n\), so \(2\equiv 0\bmod n\)---a contradiction since \(n>2\). Therefore \(r\)~and~\(-r\) are distinct. Note \(r\)~satisfies the first system, and \(-r\)~satisfies the last system. Let \(s\)~be the solution to the second system. Then \(-s\)~is the solution to the third system, and \(s\)~and~\(-s\) are also distinct. If \(r\equiv s\bmod n\), then \(r\equiv -r\bmod q\), so \(2\equiv 0\bmod q\)---a contradiction since \(q>2\). Therefore \(r\)~and~\(s\) are distinct. Similarly \(r\)~and~\(-s\) are distinct since \(p>2\). Therefore \(\{\pm r,\pm s\}\) are the four square roots of~\(a\) modulo~\(n\).\qedhere
\end{enumerate}
\end{proof}

% References
\begin{thebibliography}{0}
\bibitem{hoffstein08} Hoffstein, J. and J. Pipher and J.~H. Silverman. \textit{An Introduction to Mathematical Cryptography.} Springer, 2008.
\end{thebibliography}
\end{document}
